% no thmtools

% decide if plain theorems should be numbered (theorems, lemmas, ...)
\newcommand{\plainnumbered}{0} % 1 - numbered, 0 - unnumbered
\newcounter{plain}
\setcounter{plain}{1}

% decide if vari thms should be numbered (examples ...)
\newcommand{\varinumbered}{0} % 1 - numbered, 0 - unnumbered
\newcounter{vari}
\setcounter{vari}{1}

% this command defines a command to define theorems, lemmas, etc.
% \makeatletter
\newcommand{\defineplaintheorem}[3][alertcolor]{
    \newenvironment{#2}[1][\@nil]
        {%
        \def\tmp{##1}%
        \ifthenelse{\equal{##1}{\@nil}}
            {\ifnum\plainnumbered=0 % and no numbering is wished for
                            \def\frmtitle{\color{#1}\bfseries #3.}
                        \else % if numbering is wished
                            \def\frmtitle{\color{#1}\bfseries #3
                            \arabic{plain}\stepcounter{plain}.}
                        \fi}
            {\ifnum\plainnumbered=0 % and no numbering is wished for
                            \def\frmtitle{\color{#1}\bfseries #3 \mdseries[##1]\bfseries.}
                        \else % if numbering is wished
                            \def\frmtitle{\color{#1}\bfseries #3 \arabic{plain}\stepcounter{plain} \mdseries[##1]\bfseries.}
                        \fi}
        \color{black} % dirty fix!
        \normalsize % dirty fix!
        \begin{mdframed}[
            roundcorner=2pt,
            backgroundcolor=white!90!#1,
            linecolor=#1,
            linewidth=.7pt, 
            % linewidth=0pt, 
            frametitle={\frmtitle},
            frametitlebelowskip=2pt,
            innertopmargin=2pt,
            % frametitlerule=true,
            % frametitlebackgroundcolor=#1!25!white,
            ]
        % \tolerance 2000%
        % \emergencystretch 3em%
        % \hfuzz .5\p@
        % \vfuzz\hfuzz
        }
        {
        \end{mdframed}
        }
}


\newcommand{\definevaritheorem}[3][examplecolor]{
    \newenvironment{#2}[1][\@nil]
        {
        \def\tmp{##1}%
        \ifx\tmp\@nnil % if no name of the theorem is given
            \ifnum\varinumbered=0 % and no numbering is wished for
                \def\frmtitle{\color{#1}\bfseries #3.}
            \else % if numbering is wished
                \def\frmtitle{\color{#1}\bfseries #3
                \arabic{vari}\stepcounter{vari}.}
            \fi
        \else % if theorem name is given
            \ifnum\varinumbered=0 % and no numbering is wished for
                \def\frmtitle{\color{#1}\bfseries #3 \mdseries[##1]\bfseries.}
            \else % if numbering is wished
                \def\frmtitle{\color{#1}\bfseries #3 \arabic{vari}\stepcounter{vari} \mdseries[##1]\bfseries.}
            \fi
        \fi
        \begin{mdframed}[
            roundcorner=2pt,
            backgroundcolor=white!90!#1,
            % linecolor=green,
            linewidth=0pt,
            frametitle={\frmtitle},
            frametitlebelowskip=2pt,
            innertopmargin=2pt,
            % frametitlerule=true,
            % frametitlebackgroundcolor=alertcolor!10!white,
            ]
        % \tolerance 2000%
        % \emergencystretch 3em%
        % \hfuzz .5\p@
        % \vfuzz\hfuzz
        }
        {
        \end{mdframed}
        }
}
% \makeatother






%--------- for thmtools --------
% \def \leftrightmargin {20}
%
% \makeatletter
% % a robust wrapper of \qedsymbol
% \protected\edef\xqedsymbol{\unexpanded\expandafter{\qedsymbol}}
% \makeatother
%
%
%
% % only used for remark
% \declaretheoremstyle[
% spaceabove=0pt, spacebelow=0pt,
% headfont=\normalfont\bfseries,
% notefont=\mdseries, notebraces={(}{)},
% bodyfont=\small\normalfont,
% postheadspace=1em,
% mdframed={innertopmargin=0pt,
%           innerbottommargin=0pt,
%           innerleftmargin=\leftrightmargin pt,
%           innerrightmargin=\leftrightmargin pt,
%           hidealllines=true,
%           }
% ]{standardrem}
%
% % only used for example
% \declaretheoremstyle[
% spaceabove=0pt, spacebelow=0pt,
% headfont=\normalfont\bfseries,
% notefont=\mdseries, notebraces={(}{)},
% bodyfont=\small\normalfont,
% postheadspace=1em,
% mdframed={innertopmargin=0pt,
%           innerbottommargin=0pt,
%           innerleftmargin=\leftrightmargin pt,
%           innerrightmargin=\leftrightmargin pt,
%           hidealllines=true,
%           }
% ]{standardex}
%
% % this is used for theorems, lemmas, corollarys
% \declaretheoremstyle[
% spaceabove=3pt, spacebelow=0pt,
% headfont=\color{alertcolor}\bfseries,
% notefont=\color{alertcolor}\mdseries, notebraces={[}{]},
% bodyfont=\normalfont,
% postheadspace=\newline,
% mdframed={
%             backgroundcolor=alertcolor!10!white,
%             linewidth=0pt,
%             roundcorner=3pt,
%           }
% ]{standarddef}
%
% % only used for proof
% \declaretheoremstyle[
% spaceabove=0pt, spacebelow=0pt,
% headfont=\normalfont\itshape,
% bodyfont=\normalfont,
% postheadspace=1em,
% qed=$\blacksquare$,
% mdframed={innertopmargin=0pt,
%           innerbottommargin=0pt,
%           innerleftmargin=\leftrightmargin pt,
%           innerrightmargin=\leftrightmargin pt,
%           skipabove=-10pt,
%           hidealllines=true,
%           }
% ]{standardproof}
%
% \declaretheorem[
% style=standarddef,
% refname={Theorem,Theorems},
% Refname={Theorem,Theorems}]{theorem}
%
% \declaretheorem[
% style=standarddef,
% sibling=theorem,
% refname={Lemma,Lemmas},
% Refname={Lemma,Lemmas}]{lemma}
%
% \declaretheorem[
% style=standarddef,
% sibling=theorem,
% refname={Corollary,Corollarys},
% Refname={Corollary,Corollarys}]{corollary}
%
% \declaretheorem[
% style=standarddef,
% sibling=theorem,
% refname={Definition,Definitions},
% Refname={Definition,Definitions}]{definition}
%
% \declaretheorem[
% style=standardrem,
% sibling=theorem,
% refname={Remark,Remarks},
% Refname={Remark,Remarks}]{remark}
%
% \declaretheorem[
% style=standardex,
% sibling=theorem,
% refname={Example,Examples},
% Refname={Example,Examples}]{example}
